% !TeX encoding = UTF-8

\chapter{REVISÃO BIBLIOGRÁFICA}\label{ch:rev-bibs}

Várias técnicas conjugadas à inteligência artificial foram desenvolvidas e vêm sendo aperfeiçoadas ao longo do tempo, com o objetivo de possibilitar predições de boa acuracidade, tal como o estudo realizado por \citeonline{clements} que trabalharam com predições utilizando regressão linear. Dentro dessa linha, também estão presentes as pesquisas que se basearam na utilização de RNAs, que são citadas como referência para analisar a oscilação do mercado acionário, como os estudos realizados por \citeonline{faria}, \citeonline{kara} e \citeonline{white}, além de trabalhos desenvolvidos utilizando algoritmos genéticos, como apresentado no trabalho de \citeonline{nayak}.

O primeiro modelo para previsão de preços no mercado de ações baseado em RNA, foi desenvolvido por \citeonline{white}, o autor utilizou um padrão de rede recorrente, denominado Feedforward, com o objetivo de analisar os retornos diários das ações da International Business Machines (IBM), para testar a teoria do mercado eficiente, proposta por \citeonline{fama}, retratando que a oscilação dos preços das ações seguem uma tendência aleatória. Com bons resultados obtidos através da implementação da RNA, desde então, foi potencializada a quantidade de pesquisas referente à técnica utilizada.

Um dos trabalhos pioneiros em predição no mercado acionário com o auxílio de RNAs, foi o proposto por \citeonline{kamijo}, onde utilizaram RNAs recorrentes para o reconhecimento de padrões, através dos gráficos gerados pela análise técnica da bolsa de Tóquio. O objetivo da pesquisa foi encontrar padrões classificados como possíveis indicadores de tendências para o investimento em um determinado ativo.  

Trabalhos posteriores, como o realizado por \citeonline{ding}, define a não existência de um paradigma adequado para predições com o uso de RNAs, necessitando da análise e identificação do problema para especificar o melhor modelo de arquitetura a ser aplicado. \citeonline{ding} também evidenciam a complexidade em especificar parâmetros relevantes para a fase de treinamento de uma RNA, pois o mesmo impacta diretamente no resultado obtido, tornando-se uma etapa de grande relevância na construção de um modelo estável.

Em seu estudo \citeonline{guresen} descreveram a utilização das RNAs como uma das melhores técnicas para modelar o mercado de ações, porque as mesmas podem ser facilmente adaptadas às mudanças do mercado acionário. Em seu trabalho, os autores utilizam uma rede Multilayer Perceptron (MLP) com um total de 80 neurônios, obtendo 20 para a camada de entrada, 40 para a camada oculta, e 20 para a camada de saída, utilizando o algoritmo Backpropagation como forma de treinamento, aplicada para prever os valores dos índices das bolsas de valores americanas.

A utilização de RNAs utilizando séries temporais, tem como referência o trabalho realizado por \citeonline{waibel} que evidencia duas abordagens para análise de séries temporais: Dimensionamento temporal e RNAs recorrentes. Em sua pesquisa, foi abordada a capacidade de generalização da RNA sobre um ambiente desconhecido, podendo auxiliar na resolução dos problemas que contam com diversos padrões distintos e com alta complexidade computacional. \citeonline{waibel} utilizou uma rede Feedforward para testar predições que obtivessem resultados relevantes e satisfatórios, utilizando um padrão de previsão denominado One Step Ahead. Este padrão trabalha, segundo o autor, com uma distribuição de  N variáveis de entradas, através de uma alimentação iterativa até o momento T, onde a unidade de saída é representada por T + 1.

\citeonline{zhang} também destacam modelos de previsões através de RNAs. O trabalho enfatizou a realização de múltiplas previsões futuras, denominado Multiple Step Ahead. Os autores descrevem dois métodos apresentados na literatura para  este padrão, previsão iterativa e método direto. No primeiro método, os  valores de predição são usados iterativamente, como material para previsões futuras. Neste caso, apenas um neurônio de saída é necessário. O segundo método, consiste em colocar várias saídas na RNA, correspondente ao resultado esperado da previsão, tendo maior  relevância e utilidade  para ambientes específicos e controlados.

Trabalhos recentes utilizando métodos híbridos também vêm sendo dedicados para melhorar a capacidade de modelagem sobre séries temporais. Na pesquisa produzida por \citeonline{khandelwal} foi utilizado uma abordagem que cruza características do modelo matemático ARIMA com o de uma RNA. Os autores testaram os métodos ARIMA e RNA para separar componentes lineares e não-lineares, respectivamente, de séries temporais em valores econômicos, com o objetivo de encontrar um melhor padrão para tratar a oscilação dos dados obtidos.

Outra abordagem híbrida conhecida na literatura por ter resultados significativos juntamente com as RNAs, são os Mapas Cognitivos Fuzzy (FCMs). No trabalho realizado por \citeonline{lu} por exemplo, é demonstrado o uso de um FCM trabalhando como um modelo de RNA recorrente, capacitando o protótipo a operar através de aprendizado, integrando assim as características de ambas as técnicas aplicadas.

É importante salientar que os trabalhos citados neste capítulo, por alguma peculiaridade ou detalhe técnico serviu como motivação para a idealização deste estudo, não podendo então deixar de citá-los. Além disso, provaram ser referência dentro de suas respectivas problemáticas, trazendo análises e conclusões significativas, que são essenciais para o aperfeiçoamento acadêmico e científico.

Portanto é almejado nos capítulos posteriores, adquirir conhecimento teórico suficiente que possibilite a realização de um trabalho que  traga uma contribuição científica relevante, e que sirva como referência para os estudos futuros, como os aqui citados.