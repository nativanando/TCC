% !TeX encoding = UTF-8

\chapter{CONCLUSÕES E SUGESTÕES PARA FUTUROS TRABALHOS}\label{ch:conclusao}
O intuito dessa dissertação foi utilizar um modelo de rede neural artificial e aplica-lá para predições de cotações futuras na bolsa de valores NASDAQ. As empresas utilizadas foram a Amazon Inc, Apple Inc, Intel Corporation e Microsoft Corporation, todas caracterizadas por serem empresas de tecnologia e de grande potencial dentro da bolsa de valores.

Para realizar a especificação de uma arquitetura ideal de RNA, foi elaborada uma pesquisa sobre conceitos fundamentais que envolvem a utilização desta técnica. Durante o estudo da literatura foram identificados tópicos essenciais para adquirir um conhecimento teórico e realizar o desenvolvimento da rede, tais como topologias, algoritmos de treinamento, arquiteturas e aplicabilidade dentro do escopo do trabalho, que são modelos de redes baseadas em séries temporais. Assim, chegou-se a definição e utilização do modelo \textit{Multilayer Perceptron}, com uma arquitetura \textit{Feedforward}, 8 variáveis para a camada de entrada, 13 neurônios na camada oculta e 1 valor na camada de saída, caracterizado pelo valor da predição. O algoritmo de treinamento utilizado foi o \textit{Backpropagation} 

A partir disso foram especificadas as ferramentas para realizar a coleta dos dados, assim como para a implementação da rede neural. A linguagem de programação Python vêm se destacando dentro da área de inteligência artificial, proporcionando bibliotecas e APIs robustas para facilitar a implementação de soluções dentro desse contexto. Portanto, é importante evidenciar a utilização das bibliotecas PyBrain, pandas e matplotlib, que foram de grande auxilio para a execução da aquisição, visualização, manipulação e refinamento realizado nas séries temporais, bem como proporcionaram uma maneira simples de se trabalhar com a implementação do modelo da rede neural.

Decorrente a implementação e os resultados alcançados, pode-se evidenciar que o modelo e a arquitetura proposta da RNA obtiveram excelentes resultados em prol do objetivo principal do trabalho, que é medir a capacidade de precisão de acerto no valor de abertura das ações. A partir disso, algumas considerações gerais são necessárias para detalhar, de maneira mais precisa, o comportamento do modelo de rede implementada neste trabalho.

Portanto, detalhando todos os cenários analisados, pode-se chegar a definição de que a rede apresentou um padrão no decaimento do EQM, onde, as empresas que mais oscilaram seus valores durante a série treinada obtiveram um erro menor e precisaram de um número maior de iterações para convergir. 

Pontuando esta particularidade, é possível afirmar que o modelo trabalha com erros maiores em séries temporais onde os valores possuem pouca variação, pois os mesmos já são suficientes para chegarem em resultados próximos aos esperados. Um exemplo deste caso pode ser observado analisando os erros obtidos pela ação da Intel Corporation, que foi a rede que menos oscilou dentre as utilizadas. Portanto, pode-se concluir que em séries temporais com esta característica, os pesos dos neurônios da camada oculta da rede não requerem erros muito baixos para realizar o mapeamento e o aprendizado dos dados. 

Entretanto, em séries temporais com uma variabilidade maior nas amostras o erro resultante foi mais baixo, pois a rede necessita de um valor menor para realizar aproximações aos respectivos resultados. Este comportamento fica explícito ao visualizar as ações da Amazon, que foi a série que mais variou dentre as utilizadas neste trabalho. Portanto, pode-se concluir que em séries com esta característica, os pesos dos neurônios da camada oculta da RNA requerem erros mais baixos para realizar o mapeamento e o aprendizado dos dados.

Abordando o tema do trabalho em um contexto geral, é interessante pontuar, como sugestões para trabalhos futuros, a utilização de redes neurais que trabalham com séries temporais adjunto a outra técnica de inteligência artificial, tal como algoritmos genéticos, proporcionando uma arquitetura híbrida para a tentativa de predições. 

Também recomenda-se o desenvolvimento de trabalhos avaliando outras topologias e arquiteturas de RNAs, fomentando uma arquitetura ideal para o problema proposto. Em relação a modelagem dos dados, é interessante realizar uma pesquisa mais detalhada sobre fatores internos e externos que influenciam nos resultados das ações, alimentando assim a rede com informações mais relevantes e correlacionadas.

Para o contexto específico do trabalho, pode-se sugerir como pesquisas futuras a implementação de métodos que atuem no reconhecimento de alta e queda das ações, que foi um ponto onde a rede obteve seus maiores erros. Portanto, técnicas de ajustes e adaptabilidade para essas dispersões nos dados seriam de grande importância e agregariam um valor considerável ao modelo aplicado.