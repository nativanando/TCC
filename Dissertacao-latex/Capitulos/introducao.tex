%!TeX encoding = UTF-8

\chapter{INTRODUÇÃO}\label{ch:introducao}

A informatização do mercado acionário, que permite a movimentação de compra e venda de ações de forma eletrônica e automática, tornou-se fundamental ao longo das últimas décadas, gerando uma série de mudanças na forma em que as negociações são realizadas, se comparado ao modelo de negociação anterior, onde as movimentações aconteciam em uma unidade central, com a presença física dos investidores. Portanto, esse novo modelo mudou a forma de atuação nas bolsas de valores, possibilitando a movimentação de capitais \textit{online}, facilitando e aumentando a gama de possibilidades para a aquisição de novos acionistas. Atualmente, os papéis negociados mais relevantes em bolsas de valores, vêm de grandes empresas que dominam o mercado em suas respectivas áreas, gerando uma alta movimentação financeira que pode conceber um impacto tanto positivo como negativo no setor econômico, influenciando em decisões políticas e sociais que afetam não só a esfera econômica, mas também todos os níveis da sociedade \cite{shiller}.

Em contrapartida, houve, automaticamente, um impacto computacional significativo sobre esse novo paradigma do mercado de ações. Com o advento da Internet e o fácil acesso aos dados históricos dos papéis negociados nas bolsas de valores, através de \textit{sites} que disponibilizam essas informações, técnicas computacionais têm se tornado uma grande aliada para o diagnóstico dos preços de ações e índices, aumentando significamente as pesquisas e os modelos computacionais que possam auxiliar no aperfeiçoamento e análise do mercado financeiro.

A complexidade em prever valores futuros no mercado de ações, é um estímulo consideravelmente grande para a comunidade científica internacional, tendo em vista a diminuição do risco de investimento em ativos financeiros. Nesse contexto, técnicas como aprendizado de máquina, Redes Neurais Artificiais (RNAs) e outras soluções de inteligência artificial, vêm sendo aplicadas arduamente para possibilitar um aumento significativo na predição de valores no mercado acionário \cite{gambogi}.

Dessa maneira, uma abordagem que vem ganhando destaque, dentre as pesquisas referentes à predição de cotações futuras que utilizam séries temporais, são as RNAs. A capacidade destas técnicas em trabalhar com uma quantidade significativa de variáveis simultâneas, além da composição de sua estrutura maciçamente paralela e distribuída, evidenciam sua alta escala de poder computacional, concedendo-as habilidades de aprendizado e generalização de funções. Estas duas capacidades de processamento de informação, tornam possíveis para as RNAs resolver problemas de grande escala que, usando o processamento digital convencional, são consideradas computacionalmente inviáveis e intratáveis \cite{elpink, haykin2000}.

Neste sentido, tendo em vista a importância das redes neurais aplicadas na predição de cotações futuras de ações em bolsas de valores, o presente trabalho apresenta um estudo da aplicação destas técnicas na previsão de preços futuros dos mais relevantes papéis da NASDAQ \textit{Stock Market}, o segundo maior mercado de ações do mundo, palco onde as maiores empresas de tecnologias investem e controlam seus ativos financeiros.


\section{JUSTIFICATIVA}\label{sec:justificativa}

São vários os aspectos que impulsionam a necessidade de previsão dos índices acionários no mercado de investimentos. De fato, faz-se o principal motivo, garantir maior lucratividade possível com a realização de bons investimento, fazendo com que os investidores obtenham sempre uma vantagem sobre os demais que atuam no mercado financeiro.

Outro fator motivacional para a realização deste trabalho, utilizando um modelo de RNA, se dá pelo crescimento desta técnica aplicada a importantes áreas do contexto social. Além de sua aplicabilidade na área financeira, ferramentas tecnológicas que implementam estas técnicas têm-se feito úteis em áreas como: recursos humanos, \textit{marketing}, medicina, engenharia, dentre outras. Na área da medicina, por exemplo, é válido citar sua utilidade no diagnóstico e, até mesmo, na prevenção de futuras doenças e patologias \cite{marangoni}.

Um outro ponto que deve ser destacado e que influenciou na decisão de realizar este trabalho aplicado ao mercado financeiro, é a distribuição do modelo de dados em que as ações operam, através de oscilações diárias, formando assim, séries temporais. Tendo isso em vista, este cenário do mercado de acões proporciona uma excelente diretriz para o aperfeiçoamento de arquiteturas de RNAs que utilizam um modelo de dados baseado em séries temporais.

Por fim, o vasto crescimento de aplicações que utilizam o auxílio de tecnologias baseadas em técnicas de inteligência artificial também foi um fator determinante para definir a aplicabilidade deste trabalho. Sistemas inteligentes com apoio a tomada de decisões em tempo real e análises complexas de grandes quantidades de dados estão cada vez mais presentes no cotidiano das pessoas.

\section{OBJETIVOS}\label{sec:objetivos}
A partir do tema definido, estabelecem-se os objetivos a serem alcançados ao término deste trabalho. Assim temos, respectivamente, o objetivo principal e os objetivos específicos, apresentados em ordem lógica de desenvolvimento.
\subsection{Objetivo Geral} 
Atribuiu-se, como objetivo principal deste trabalho, especificar e aplicar um modelo de RNA para realizar predições nos valores de abertura das ações na bolsa de valores NASDAQ.

\subsection{Objetivos Específicos}\label{subsec:objetivos_especificos}
Com o objetivo geral determinado, tornaram-se necessários as definições de alguns objetivos específicos, sendo:
\begin{itemize}
	\item Especificar uma arquitetura de RNA;
	\item Definir um ambiente para o desenvolvimento da RNA;
	\item Utilizar uma Interface de programação de aplicações (API) para a coleta dos dados que serão analisados;
	\item Treinar o modelo de RNA especificado;
	\item Realizar testes com o modelo treinado, através das ações utilizadas;
	\item Analisar os resultados obtidos pela implementação, comparando-os com os seus resultados reais e avaliando sua capacidade de precisão.
\end{itemize}

\section{ESTRUTURA DO DOCUMENTO}\label{sec:organizacao-trabalho}
Este documento será organizado em sete capítulos: introdução, revisão bibliográfica, fundamentação teórica, materiais e métodos, implementação, resultados, conclusões e referências bibliográficas.

O presente capítulo contextualiza e introduz o assunto a ser tratado, bem como define os objetivos a serem alcançados.

No segundo capítulo, são apresentados os trabalhos que foram utilizados como referência e motivação, além de demonstrar suas contribuições científicas.

A fundamentação teórica apresentada no terceiro capítulo, descreve o embasamento referente aos conceitos das técnicas que serão utilizadas para a realização deste estudo.

O quarto capítulo é destinado ao detalhamento de métodos, técnicas e procedimentos que têm por objetivo gerar conhecimento para viabilizar um melhor ambiente de desenvolvimento para o trabalho.

O quinto capítulo detalha a implementação dos métodos, que segue as especificações das técnicas levantadas no capítulo anterior.

No sexto capítulo são expostos os experimentos realizados sobre o modelo implementado, destacando sua capacidade de precisão aplicada ao objetivo do trabalho.

Por fim, no sétimo capítulo são tratadas as conclusões obtidas pela técnica de RNA desenvolvida, destacando suas vantagens e desvantagens. Também serão idealizadas alternativas para trabalhos futuros, os quais dariam continuidade e complementariam a pesquisa realizada.