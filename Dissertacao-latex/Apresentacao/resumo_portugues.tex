% ---
% RESUMO PORTUGUÊS
% ---
\begin{resumo}[RESUMO]	
A predição no mercado eletrônico financeiro é um desafio considerável para proporcionar aos investidores uma boa capacidade de atuação no processo de compra e venda de ações. Tendo isso em vista, técnicas computacionais vêm sendo aplicadas arduamente para realizar análises dentro deste cenário de atuação. Dentre as técnicas, as Redes Neurais Artificiais (RNAs), uma subárea da Inteligência Artificial, que busca simular computacionalmente a forma de processamento do cerébro humano, vem ganhando destaque para este propósito. Sendo assim, este trabalho apresenta a especificação e aplicabilidade de um modelo de RNA para realizar predições no valor de abertura das ações na bolsa de valores NASDAQ. Além disso, são evidenciadas e detalhadas ferramentas que auxiliam o desenvolvimento dessa técnica de forma eficiente, utilizando a linguagem de programação Python e suas bibliotecas para análise de dados.

\vspace{\onelineskip}
    
\noindent
\textbf{Palavras-chaves}: Redes Neurais Artificiais. Inteligência Artificial. Mercado Financeiro. Python.
 % 4 palavras separadas por . (ponto)
\end{resumo}
